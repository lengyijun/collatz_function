%tpm601.tex Simulation of the 3x+1 function, Version 1

\documentclass[10pt]{article}
\usepackage{amssymb}
\usepackage{xcolor}

\addtolength{\oddsidemargin}{-1cm}
\addtolength{\textwidth}{2cm}

\newtheorem{prop}{Proposition}[section]
\newtheorem{thm}[prop]{Theorem}
\newtheorem{lem}[prop]{Lemma}
\newtheorem{corol}[prop]{Corollary}
\newtheorem{defi}[prop]{Definition}
\newtheorem{fact}[prop]{Fact}

\newcommand{\bin}{\mathop{\mbox{bin}}\nolimits}

\date{Feburary 23, 2025}

\begin{document}

This tex copy from \cite{Mi14}, thanks!

Preliminaries and background: see \cite{Mi14}.

\begin{table}
$$\begin{array}{c|c|c|c|c|c|c|c|c|c|c|c|c|c}
\mbox{symbols} & \multicolumn{13}{c}{}\\                                                \cline{1-3}
10             & \textcolor{green}{Ma}    & \mbox{\bf Mi}_2& \multicolumn{11}{c}{}\\                       \cline{1-3}
9              &       &      &\multicolumn{11}{c}{}\\                                  \cline{1-3}
8              & \textcolor{green}{Ba}    &      &\multicolumn{11}{c}{}\\                                  \cline{1-3}
7              &       & Da   			 &\multicolumn{11}{c}{}\\                                  \cline{1-4}
6              &       & \textcolor{green}{Ma}   &  \mbox{\bf Mi}_2 & \multicolumn{10}{c}{}\\          \cline{1-4}
5              &       & \textcolor{green}{Ba}   &                  & \multicolumn{10}{c}{}\\          \cline{1-5}
4              &       & \textcolor{red}{Mi_2} & \textcolor{green}{Ma}               &\mbox{\bf Mi}_2 & \multicolumn{9}{c}{}\\         \cline{1-6}
3              &       &      &                  & \textcolor{blue}{Ma}   &\mbox{\bf Mi}_1 & \multicolumn{8}{c}{}\\  \cline{1-13}
2              &       &      &    &      &      &     &     & \textcolor{green}{Yj}  & \textcolor{green}{Ba}  & \textcolor{blue}{Ma} & & \mbox{\bf Mi}_2\\  \hline
               & 2     & 3    & 4  & 5    &\,6\, &\,7\,&\,8\,&\,9\,& 10  & 11 & 12 & 13 & \mbox{states}
\end{array}$$
\caption{Turing machines simulating the $3x + 1$ function:
$Ma=$ Margenstern \cite{Ma98,Ma00},
$Ba=$ Baiocchi \cite{Ba98},
$Mi_1=$ Michel \cite{Mi93},
$Mi_2=$ Michel \cite{Mi14}.
$Da=$ Daniel \cite{Da24}.
$Yj=$ Yijun Leng (this repo).
In roman boldface, halting machines.
Green: unary;
Blue: base 2;
Red: base 3;
}
\end{table}



\begin{thebibliography}{99}
\bibitem{Ba98} C.\ Baiocchi, 3N+1, UTM e Tag-systems (Italian),
Dipartimento di Matematica dell'Universit\`a ``La Sapienza'' di Roma {\bf 98/38}, 1998.

\bibitem{BM01} C.\ Baiocchi and M.\ Margenstern, Cellular automata about
the $3x + 1$ problem, in: Proc.\ LCCS'2001, Universit\'e Paris 12, 2001, 37--45,
available on the website http://lacl.univ-paris12.fr/LCCS2001/.

\bibitem{La10} J.C.\ Lagarias (Ed.), The Ultimate Challenge: The 3$x$+1 Problem, AMS, 2010.

\bibitem{Ma98} M.\ Margenstern, Frontier between decidability and undecidability: a survey,
Proc.\ MCU'98, Vol.\ 1, ISBN 2-9511539-2-9, 1998, 141--177.

\bibitem{Ma00} M.\ Margenstern, Frontier between decidability and undecidability:
a survey, \emph{Theoret.\ Comput.\ Sci.} {\bf 231}, 2000, 217--251.

\bibitem{Mi93} P.\ Michel, Busy beaver competition and Collatz-like problems,
\emph{Arch.\ Math.\ Logic} {\bf 32} (5), 1993, 351--367.

\bibitem{Mi14} P.\ Michel, Simulation of the Collatz 3x+ 1 function by Turing machines. arXiv preprint arXiv:1409.7322. 2014 Sep 25.

\bibitem{MM10} P.\ Michel and M.\ Margenstern, Generalized 3$x$+1 functions and the theory of computation,
in \cite{La10}, 105--128.

\bibitem{WN09} D.\ Woods and T.\ Neary, The complexity of small universal Turing machines:
a survey, \emph{Theoret. Comput. Sci.} {\bf 410}, 2009, 443--450.
Extended and updated in: http://arxiv/abs/1110.2230.

\bibitem{Da24} Daniel, https://discord.com/channels/960643023006490684/1084047886494470185/1305239579749646520

\end{thebibliography}












\end{document}
